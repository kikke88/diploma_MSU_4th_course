%Разработка метода прогнозирования слабой масштабируемости суперкомпьютерных приложений
\section{Постановка задачи}
	Целью данной работы является разработка метода прогнозирования слабой масштабируемости суперкомпьютерных приложений. На разрабатываемый метод накладываются условия универсальности, то есть для построения прогноза не должен использоваться или как-либо модифицироваться исходный код программ и не должны использоваться сведения о вычислительной системе, на которой производятся запуски.

	Для достижения данной цели требуется решить следующие задачи:
	\begin{itemize}
		\item Разработать метод, предсказывающий слабую масштабируемость суперкомпьютерных приложений на основе экспериментальных данных.
		\item Проверить применимость метода на различных приложениях, собрав экспериментальную базу и оценив точность предсказаний.
		\item ???Разработать программное средство, позволяющее строить прогноз слабой масштабируемости приложения в соответствии с разработанным методом.(ЕЩЁ ПОДУМАТЬ НАД ФОРМУЛИРОВКОЙ и вообще оставлять ли этот пункт)???
	\end{itemize}


\clearpage
% \newpage
%На счёт же третьей постановки - если мы ставим задачу разработки программного средства, то это средство нужно будет предъявить, и к нему будут предъявлять все обычные требования как к программному средству по оформлению, документированию и т.д., об этом тогда тоже не забудьте позаботиться.