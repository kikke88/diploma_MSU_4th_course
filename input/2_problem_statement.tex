%Разработка метода прогнозирования слабой масштабируемости суперкомпьютерных приложений
\section{Постановка задачи}
	Целью данной работы является разработка метода прогнозирования слабой масштабируемости суперкомпьютерных приложений. Метод должен удовлетворять требованиям универсальности, то есть для своей работы он не должен использовать информацию о коде, алгоритме и системе, на которой производятся запуски.
	% На разрабатываемый метод накладываются условия универсальности, то есть для своей работы он не должен использовать информацию о коде, алгоритме и системе, на которой производятся запуски.

	Для достижения поставленной цели требуется решить следующие задачи:
	\begin{itemize}
		\item Разработать метод, предсказывающий слабую масштабируемость суперкомпьютерных приложений на основе экспериментальных данных.
		\item Проверить применимость метода на различных приложениях, собрав экспериментальную базу и оценив точность предсказаний.
	\end{itemize}

\clearpage
% \newpage
%На счёт же третьей постановки - если мы ставим задачу разработки программного средства, то это средство нужно будет предъявить, и к нему будут предъявлять все обычные требования как к программному средству по оформлению, документированию и т.д., об этом тогда тоже не забудьте позаботиться.