%Разработка метода прогнозирования слабой масштабируемости суперкомпьютерных приложений
\chapter{Постановка задачи}
	Целью данной работы является разработка метода прогнозирования слабой масштабируемости суперкомпьютерных приложений. Для построения прогноза не должены использоваться или как-либо модифицироваться исходный код программ и сведения о вычислительной системе, на которой производятся запуски.

	Для достижения данной цели требуется решить слудующие задачи:
	\begin{itemize}
		\item Разработать метод предсказывающий слабую масштабируемость суперкомпьютерных приложений на основе экспериментальных данных.
		\item Проверить применимость метода на различных приложениях, собрав экспериментальную базу и оценив точность предсказаний.
		%\item Провести эксперименты(ИЛИ собрать экспериментальную базу) и проверить применимость метода, оценить(ИЛИ оценив) точность предсказаний.
		\item ???Разработать программное средство, позволяющее строить прогноз слабой масштабируемости приложения в соответствии с разработанным методом.(ЕЩЁ ПОДУМАТЬ НАД ФОРМУЛИРОВКОЙ)???
	\end{itemize}


\clearpage
%На счёт же третьей постановки - если мы ставим задачу разработки программного средства, то это средство нужно будет предъявить, и к нему будут предъявлять все обычные требования как к программному средству по оформлению, документированию и т.д., об этом тогда тоже не забудьте позаботиться.