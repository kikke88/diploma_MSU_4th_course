\section{Список литературы}
	\begingroup
		\renewcommand{\section}[2]{}%
		\begin{thebibliography}{}
			\bibitem{top500}
			The 54nd edition of the TOP500 list [Электронный ресурс]. – Электрон. дан. – URL: https://www.top500.org/lists/2019/11/. (дата обращения 16.03.2020).

			\bibitem{Perf_low}
			Leland, Robert \& Ang, Jim \& Barnette, Daniel \& Benner, Bob \& Goudy, Sue \& Malins, Bob \& Rajan, Mahesh \& Vaughan, Courtenay \& Black, Amalia \& Doerfler, Doug \& Domino, Stefan \& Franke, Brian \& Ganti, Anand \& Laub, Tom \& Meyer, Hal \& Scott, Ryan \& Stevenson, Joel \& Sturtevant, Judy \& Taylor, Mark. (2016). Performance, Efficiency and Effectiveness of Supercomputers.

			\bibitem{scalability_def}
			Antonov, Alexander \& Teplov, Alexey. (2016). Generalized Approach to Scalability Analysis of Parallel Applications. 291-304. 10.1007/978-3-319-49956-7\_23. 

			\bibitem{COMPLEXITY}
			Абрамов С.А. Лекции о сложности алгоритмов - М.: МЦНМО, 2012. - 245 с.

			\bibitem{scaling_types}
			Антонов, А. С., \& Теплов, А. М. (2013). Исследование масштабируемости программ с использованием инструментов анализа параллельных приложений на примере модели атмосферы Nh3d. Вестник Южно-Уральского государственного университета. Серия: Вычислительная математика и информатика, 2 (1), 5-16.

			\bibitem{log_main}
			Barnes, B. J., Reeves, J., Rountree, B., De Supinski, B., Lowenthal, D. K., \& Schulz, M. (2008). A regression-based approach to scalability prediction. In ICS'08 - Proceedings of the 2008 ACM International Conference on Supercomputing (pp. 368-377). (Proceedings of the International Conference on Supercomputing). https://doi.org/10.1145/1375527.1375580
		
			\bibitem{efficiency_prediction} 
			Rosas, C., Giménez, J., \& Labarta, J. (2014). Scalability prediction for fundamental performance factors. Supercomputing Frontiers And Innovations, 1(2), 4-19. doi:http://dx.doi.org/10.14529/jsfi140201

			\bibitem{focused_regression}
			Barnes, Brad \& Garren, Jeonifer \& Lowenthal, David \& Reeves, Jaxk \& Supinski, Bronis \& Schulz, Martin \& Rountree, Barry. (2010). Using focused regression for accurate time-constrained scaling of scientific applications. Proceedings of the 2010 IEEE International Symposium on Parallel and Distributed Processing, IPDPS 2010. 1 - 12. 10.1109/IPDPS.2010.5470431. 

			\bibitem{analytic_func}
			Alexandru Calotoiu, Torsten Hoefler, Marius Poke, and Felix Wolf. 2013. Using automated performance modeling to find scalability bugs in complex codes. In Proceedings of the International Conference on High Performance Computing, Networking, Storage and Analysis (SC ’13). Association for Computing Machinery, New York, NY, USA, Article 45, 1–12. DOI:https://doi.org/10.1145/2503210.2503277

			\bibitem{ML_SMG2000}
			Ipek E., de Supinski B.R., Schulz M., McKee S.A. (2005) An Approach to Performance Prediction for Parallel Applications. In: Cunha J.C., Medeiros P.D. (eds) Euro-Par 2005 Parallel Processing. Euro-Par 2005. Lecture Notes in Computer Science, vol 3648. Springer, Berlin, Heidelberg

			\bibitem{ML_Grid}
			Nadeem, Farrukh \& Alghazzawi, Daniyal \& Mashat, Abdulfattah \& Fakieh, Khalid \& Almalaise, Abduallah \& Hagras, Hani. (2017). Modeling and predicting execution time of scientific workflows in the Grid using radial basis function neural network. Cluster Computing. 20. 2805–2819. 10.1007/s10586-017-1018-x. 

			\bibitem{ML_PROC_KERN}
			Singh, Karan \& Curtis-Maury, Matthew \& McKee, Sally \& Blagojevic, Filip \& Nikolopoulos, Dimitrios \& Supinski, Bronis \& Schulz, Martin. (2010). Comparing Scalability Prediction Strategies on an SMP of CMPs. 6271. 143-155. 10.1007/978-3-642-15277-1\_14.

			\bibitem{simulation_FASE}
			Grobelny, Eric \& Bueno, David \& Troxel, Ian \& George, Alan \& Vetter, Jeffrey. (2007). FASE: A Framework for Scalable Performance Prediction of HPC Systems and Applications. Simulation. 83. 10.1177/0037549707084939. 

			\bibitem{representative_replay}
			Zhai, Jidong \& Chen, Wenguang \& Zheng, Weiming \& Li, Keqin. (2015). Performance Prediction for Large-Scale Parallel Applications Using Representative Replay. IEEE Transactions on Computers. 65. 1-1. 10.1109/TC.2015.2479630. 			

			\bibitem{UV_matrix}
			Shao, Qingshi \& Li, Pan \& Liu, Shijun \& Liu, Xinyan. (2017). A collaborative filtering based approach to performance prediction for parallel applications. 331-336. 10.1109/CSCWD.2017.8066716. 

			\bibitem{Kazmina_Antonov_article}
			К.П. Казьмина, А.С. Антонов. Разработка методов прогнозирования масштабируемости приложений на конфигурации суперкомпьютеров // Вестник компьютерных и информационных технологий. N 12, 2018. С. 45-56.(http://www.vkit.ru/index.php/current-issue-rus/770-045-056) DOI:10.14489/vkit.2018.12.pp.045-056

			\bibitem{Kazminf_Valkon_Antonov_article}
			Pavel Valkov, Kristina Kazmina, and Alexander Antonov. Using Empirical Data for Scalability Analysis of Parallel Applications // Communications in Computer and Information Science. Vol. 1063. 2019. Pp. 58-73. DOI:10.1007/978-3-030-28163-2\_5

			\bibitem{Lom2_stat}
			Voevodin, V., Antonov, A., Nikitenko, D., Shvets, P., Sobolev, S., Sidorov, I., Stefanov, K., Voevodin, V., \& Zhumatiy, S. (2019). Supercomputer Lomonosov-2: Large Scale, Deep Monitoring and Fine Analytics for the User Community. Supercomputing Frontiers And Innovations, 6(2), 4-11. doi:http://dx.doi.org/10.14529/jsfi190201

			\bibitem{HPCG}
			Report, S., Dongarra, J., \& Heroux, M.A. (2013). Toward a New Metric for Ranking High Performance Computing Systems.

			\bibitem{SUMMA}
			Robert A. van de Geijn and Jerrell Watts. 1995. SUMMA: Scalable Universal Matrix Multiplication Algorithm. Technical Report. University of Texas at Austin, USA.		 
		
		\end{thebibliography}
	\endgroup