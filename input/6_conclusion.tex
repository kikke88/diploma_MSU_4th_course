\section{Заключение}
	\begin{figure}
		\centering
		\includegraphics[width=\textwidth]{./images/RE_graph}
		\caption{Относительные ошибки предсказаний по всем рассматриваемым приложениям}
		\label{RESULT}
	\end{figure}
	В данной работе были получены следующие основные результаты:
	\begin{itemize}
		\item Разработан метод, предсказывающий слабую масштабируемость суперкомпьютерных приложений на основе экспериментальных данных со средней относительной ошибкой по всем смотренным приложениям равной 8,6\%.
		%\item Выполнена проверка применимости метода на различных приложениях, с помощью запусков приложений HPL, HPCG, матричных алгоритмов умножения SUMMA и DNS, Graph500 на суперкомпьютере "<Ломоносов-2">.
		\item Выполнена экспериментальная проверка метода с помощью запусков на суперкомпьютере "<Ломоносов-2"> на примере HPL, HPCG, алгоритмов матричного умножения (SUMMA и DNS), Graph500.
	\end{itemize}

	На основании предложенного метода удалось построить предсказания слабой масштабируемости для всех рассматриваемых приложений так, что максимальная относительная ошибка среди всех приложений и конфигураций не превышает 28\%, однако подобный значения являются единичными случаями. Средние значения относительных ошибок для различных приложений равны HPL - 4,9\%, HPCG - 5,6\%, SUMMA - 3,6\%, DNS - 6,4\%, Graph500 - 13,2\%. Так как почти треть всех предсказаний значений динамических характеристик приходится на Graph500, а на этом приложении сильнее всего отразились трудности заданием конфигураций запусков, поэтому  относительные ошибки именно на этом приложении во многом являются определяющими, при расчёте среднего значения ошибок по всем приложениям, которое составляет 8,6\%. Гистограмма с различными значениями относительных ошибок представлена на рисунке \eqref{RESULT}. Таким образом, относительные ошибки предсказания предложенного метода
	%предложенный метод предсказания даёт относительные ошибки, сравнимые 
	сравнимы с ошибками предсказания существующих подходов при сопоставимых размерах конфигураций предсказываемых запусков. Но разработанный метод отличается от них простотой построения, отсутствием необходимости собирать большой набор тестовых данных и способностью строить предсказания, не используя информацию о коде, алгоритме и системе, на которой производятся запуски, то есть он является достаточно универсальным.
	%Помимо этого, он удовлетворяет поставленным условиям универсальности.
	
\clearpage
%\newpage