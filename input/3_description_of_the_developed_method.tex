
\chapter{Описания разработанного метода}
	%При изучении ли масштабируемости разрабатываемого приложения или поиске оптимальной конфигурации запуска, такой что 
	%- эмпирические данные
	В работе рассматривается слабая масштабируемость суперкомпьютеных приложений, которая характеризует способность параллельной программы сохранять эффективность распараллеливания при увеличении числа процессоров и одновременном сохранении объема работы, приходящейся на каждый процессор, то есть
	\begin{equation}\label{weak_sc}
	T_A(N)\:/\:p = const
	\end{equation}
	Для того чтобы исследовать слабую масштабируемость должно быть заранее известно, какие параметры запуска отвечают за определение размера задачи и как именно временная сложность работы программы зависит от этих самых параметров.

	В качестве метода для построения предсказания из всех рассмотренных ранее выбрана линейная регрессия%основы для построения модели предсказаний среди всех рассмотренных ранее методов выбрана линейная регрессия
	, потому что, во-первых, модель получается простой, что упрощает поиск неизвестных параметров, однако точность предсказаний либо сопоставима, либо даже лучше, чем у остальных методов, во-вторых, чтобы выполнить поиск регрессионных параметров не требует большого количества запусков приложения, как, например, для обучения нейронной сети или построения прогноза с помощью коллаборативной фильтрации, в-третьих, для построения точного прогноза нет необходимости использовать исходный код программ и сведения о вычислительной системе, то есть выбранный метод является универсальным.%удовлетворяет всем поставленным требованиям универсальности.

	С помощью линейной регрессии строятся предсказания значений динамических характеристик работы программы на \(p\) процессах, где \[p \in \mathbb{P} = \{p_1, p_2, \ldots, p_m \},\ p_1 < p_2 < \ldots < p_m\] Для предсказания используются эмпирические данные, полученные из несколько запусков этой же программы на \(q\) процессах, где \[q \in \mathbb{Q} = \{q_1,\ldots, q_n\},\ q_1 < q_2 < \ldots < q_n < p_1\] Стоит отметить, что из-за того, что рассматривается слабая масштабируемость, то зная количество процессов и используя соотношение \eqref{weak_sc}, можно установить сложность задачи, а по ней определить необходимые для сохранения соотношения параметры запуска программы. Тогда процессы из множества \(\mathbb{P}\) вместе с соответствующими им параметрами запуска будем называть целевыми конфигурациями, аналогично для множества \(\mathbb{Q}\) - тестовыми конфигурациями.
	%Количество процессов вместе с этими параметрами запуска будем называть целевой конфигурацией 

	Предиктор значения динамической характеристики представляет собой функцию, зависящую от параметров запуска, определяющих конфигурацию, \((x_1, x_2, \ldots, x_n)\), таких как размер задачи, конфигурация процессорной сетки, размеры задачи обсчитываемой локально одним процессом и другие, и от количества используемых процессов \(q\):
	\begin{equation}\label{main_formula}
	DF = \hat{DF} + error = F(x_1, x_2, \ldots, x_n, q) + error
	\end{equation}
	Где \(DF\ и\ \hat{DF}\) - полученное из эксперимента и предсказываемое значения динамической характеристики. Предполагается, что функция \(F\) зависит от неизвестных регрессионных параметров линейно.

	%как именно будут изменяться различные динамические характеристики приложения при увеличении числа используемых процессов в условиях слабой масштабируемости.

	%Основной интерес представляют конфигурации приложения, использующие большое количество системных ресурсов, прежде всего речь идёт о вычислительных узлах. 
	
%-----------------------------------------------
	% D - слабая масштабируемость, какие параметры меняются
	% - сложно получить много ресурсов
	% D - Использование запуском на конфигурациях малого размера и экстаполяция результатов на большие запуски(ссылка на такое же в статьях)
	% D - тестовые(малые)/целевые(большие) запуски(конфигурации)
	% - конфигурация = параметры запуска
	% - предсказывается только одна динамическая характеристика
%-----------------------------------------------

	Для нахождения оптимальных значений этих параметров, используется метод наименьших квадратов, он минимизирет сумму квадратов ошибок \(SSE = |f(w, g(x)) - y|_2 = \sum_{i = 1}^{N}{(f(w, g_i(x)) - y_i)^2 \rightarrow min}\), что эквивалентно минимизации абсолютной ошибки \(error\) из формулы \eqref{main_formula}.
	
	Однако для оценки качества предсказаний принято использовать относительную ошибку, которая вычисляется по формуле \ref{RE}. Если использовать введённые ранее обозначения, то её можно переписать в виде \ref{RE_norm}. Получается, что относительная ошибка зависит не только от минимизируемых с помощью метода наименьших квадратов ошибок, но и от значений динамических характеристик, из-за этого на малом количестве процессов абсолютная ошибка может быть маленькой при большой относительной, а на большой количестве процессов большой при малой относительной ошибке, что никак не могло устраивать. Поэтому исходное выражение необходимо было модифицировать. Для этобы был применён такой же, как и в \cite{log_main}, подход, а именно приближениние аппроксимируемой величины в логарифмическом масштабе. Тогда исходное выражение \ref{main_formula} преобразуется в \eqref{main_log_formula}.
	\begin{equation}
		\label{RE} 
		RE = \frac{|DF - \hat{DF}|}{DF}
	\end{equation}
	\begin{equation}
		\label{RE_norm}
		RE_{norm} = \frac{|error|}{DF}
	\end{equation}
	А формула \ref{RE_norm} для вычисления относительной ошибки в \ref{RE_log}. Теперь значение относительной ошибки зависит только от минимизуруемой абсолютной ошибки, что и являлось целью преобразования.
	\begin{equation}\label{main_log_formula}
	\log{(DF)} = \log{\hat{(DF)}} + error
	\end{equation}
	\begin{equation}
		\label{RE_log}
		RE_{log}= 2^{|error|} - 1
	\end{equation}
	Ключевым шагом является параметризация функции \(log(DF) = log(F(x_1, x_2, \ldots, x_n, q))\). Наиболее точные предсказания удалось получить при помощи функции:
	\begin{equation}
	\log{(DF)} = \beta_{1} \cdot \log{(q)} + \beta_{2} \cdot \log{(N)} + \beta_{3} \cdot \log{(q)} \cdot \log{(N)} + error 
	\end{equation}
	Здесь за \(N\) обозначен размер задачи, а за \(q\) количество процессов, на которых эта задача запускается. Несмотря на наличие логарифмов, статистически это всё ещё линейная модель, поскольку она линейна относительно неизвестных параметров, поэтому можно использовать обширную статистическую теорию линейных моделей, в том числе поиск коэффициента с помощью метода наименьших квадратов.


	Для нескольких запусков приложения с фиксированными параметрами запуска наблюдается разброс наблюдаемых значений динамических характеристик. Это можно объяснить различным размещением процессов на узлах вычислительной системы, различной степерью загруженности коммуникационной сети во время работы приложения(мб написать:, изношенностью компонент вычислительных узлов). Поэтому необходимо проводить множественные(5-7) запуски приложения в идентичных конфигурациях, а затем для каждой конфигурации, в зависимости от рассматриваемой динамической характеристики, выбирать достигнутый максимум/минимум значения динамической характеристики среди запусков с фиксированными параметрами запуска.

	В качестве динамических характеристик, для которых строится предсказание масштабируемости, выбраны время выполниния программы и её производительность. Для двух из пяти рассматриваемых программ производительность измеряется в GFlops, ещё для одной в MTeps, а для остальных двух рассматривается только время выполнения, так как точно посчитать производительность не представляется возможным. Для набора запусков с идентичными конфигурациями выбирается минимум времени исполнения и максимум производительности среди всех запусков набора.


	Общая схема работы метода:
	\begin{enumerate}[I]
	\item Проведение запусков с тестовыми конфигурациями.
	\item Извлечение из результатов запусков необходимые для поиска неизвестных коэффициентов данные, для идентичных конфигураций выбирается минимум времени / максимум производительности исполнения.
	\item Использование метода наименьших квадратов для подбора коэффициентов линейной регрессии.
	\item Постноение предсказаний значения динамической характеристики для заданного множества целевых запусков с помощью построенной модели.
	\end{enumerate}
% 	\section{Общая схема работы метода}
% \clearpage