\begin{thebibliography}{00}

	\bibitem{efficiency_prediction} 
	Rosas C., Giménez J., Labarta J. Scalability prediction for fundamental
	performance factors. Supercomputing Frontiers and Innovations, Vol 1, No 2,
	2014, p. 4-19

	\bibitem{simulation_FASE}
	FASE: A Framework for Scalable Performance Prediction of HPC Systems and Applications
	/ Grobelny E., Bueno D., Troxel I. et al. // SIMULATION. 2007. Vol. 83, No. 10. P. 721-745.

	\bibitem{representative_replay}
	Zhai J., Chen W., Zheng W., Li K. Performance Prediction for Large-Scale Parallel
	Applications Using Representative Replay // IEEE Transactions on Computers. 2016. Vol. 65,
	No. 7. P. 2184-2198.

	\bibitem{analytic_func}
	A. Calotoiu, T. Hoefler, M. Poke \& F. Wolf, "Using automated performance modeling
	to find scalability bugs in complex codes," SC '13: Proceedings of the International
	Conference on High Performance Computing, Networking, Storage and Analysis, Denver,
	CO, 2013, pp. 1-12

	\bibitem{log_main}
	A regression-based approach to scalability prediction / Barnes B. J., Rountree B., Lowenthal
	D. K. et al. // Proceedings of the ICS. 2008. P. 368-377.

	\bibitem{focused_regression}
	B. Barnes et al., "Using focused regression for accurate time-constrained scaling of scientific applications," 2010 IEEE International Symposium on Parallel and Distributed Processing (IPDPS), Atlanta, GA, 2010, pp. 1-12.

	\bibitem{UV_matrix}
	Q. Shao, L. Pan, S. Liu \& X. Liu, "A collaborative filtering based approach to performance prediction for parallel applications," 2017 IEEE 21st International Conference on Computer Supported Cooperative Work in Design (CSCWD), Wellington, 2017, pp. 331-336.

	\bibitem{ML_SMG2000}
	Ipek E., de Supinski B., Schulz M., McKee S. A. An approach to performance prediction for parallel applications // Euro-Par Parallel Processing. Lecture Notes in Computer Science. 2005. Vol. 3648. P. 196-205.

	\bibitem{ML_Grid}
	Nadeem, F., Alghazzawi, D., Mashat, A. et al. Modeling and predicting execution time of scientific workflows in the Grid using radial basis function neural network. Cluster Comput 20, 2805–2819 (2017).

	\bibitem{ML_PROC_KERN}
	Singh K. et al. (2010) Comparing Scalability Prediction Strategies on an SMP of CMPs. In: D’Ambra P., Guarracino M., Talia D. (eds) Euro-Par 2010 - Parallel Processing. Euro-Par 2010. Lecture Notes in Computer Science, vol 6271. Springer, Berlin, Heidelberg

	\bibitem{scaling_types}
	Антонов, А. С., Теплов, А. М. (2013). Исследование масштабируемости программ с использованием инструментов анализа параллельных приложений на примере модели атмосферы Nh3d. Вестник Южно-Уральского государственного университета. Серия: Вычислительная математика и информатика, 2 (1), 5-16.

	\bibitem{scalability_def}
	Alexander Antonov, Alexey Teplov. Generalized Approach to Scalability Analysis of Parallel Applications // Lecture Notes in Computer Science. Vol.10049, 2016. Pp. 291-304. DOI: 10.1007/978-3-319-49956-7\_23

  	\bibitem{top500}
  	The 54nd edition of the TOP500 list [Электронный ресурс]. – Электрон. дан. – URL: https://www.top500.org/lists/2019/11/. (дата обращения 16.03.2020).

	\bibitem{Kazmina_Antonov_article}
	К.П. Казьмина, А.С. Антонов. Разработка методов прогнозирования масштабируемости приложений на конфигурации суперкомпьютеров // Вестник компьютерных и информационных технологий. N 12, 2018. С. 45-56.(http://www.vkit.ru/index.php/current-issue-rus/770-045-056) DOI:10.14489/vkit.2018.12.pp.045-056

  	\bibitem{Kazminf_Valkon_Antonov_article}
    Pavel Valkov, Kristina Kazmina, and Alexander Antonov. Using Empirical Data for Scalability Analysis of Parallel Applications // Communications in Computer and Information Science. Vol. 1063. 2019. Pp. 58-73. DOI:10.1007/978-3-030-28163-2\_5

    \bibitem{Lom2_stat}
	Voevodin, V., Antonov, A., Nikitenko, D., Shvets, P., Sobolev, S., Sidorov, I., Stefanov, K., Voevodin, V., \& Zhumatiy, S. (2019). Supercomputer Lomonosov-2: Large Scale, Deep Monitoring and Fine Analytics for the User Community. Supercomputing Frontiers And Innovations, 6(2), 4-11. doi:http://dx.doi.org/10.14529/jsfi190201

	\bibitem{HPCG}
	Report, S., Dongarra, J., \& Heroux, M.A. (2013). Toward a New Metric for Ranking High Performance Computing Systems.
	
	\bibitem{SUMMA}
	Robert A. van de Geijn and Jerrell Watts. 1995. SUMMA: Scalable Universal Matrix Multiplication Algorithm. Technical Report. University of Texas at Austin, USA.




\end{thebibliography}
\clearpage